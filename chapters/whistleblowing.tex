\chapter{Case: whistleblowing}\label{ch:whistleblowing}
\section{Introduction}
This chapter is structured as followed: firstly we will identify relevant stakeholders and key issues from the case. Secondly we will try to accredit some degree of responsibility to the identified stakeholders. Finally we will describe how we would bring the issue to attention and our proposed solutions.

\section{Stakeholders}
Firstly it is important to identify the relevant stakeholders for our discussion. Our primary stakeholders are parties that could be held accountable for a potential data leak. These stakeholders include individual employees (including us), the company that developed the hotel's software and the hotel(s) using the software.  The secondary stakeholders are people that are are affected by a data leak or that are otherwise involved in the process. These stakeholders are people whose personal data is leaked. Guests can include high-profile bodies e.g. privacy councils (and other relevant government agencies) and news corporations whose public image is of critical value to the organization.\\
The key issue of the case is a fault in the provided software which could lead to a potential leak of personal data. The question is: Who is legally or morally responsible and how should an employee of the software company approach the issue of mitigating or reporting the deficit in an environment unwilling to collaborate towards a solution?

\section{Responsibility}
Now that we have identified our stakeholders we can commence a discussion on where the responsibility of the product and the accompanying fault could lie. If we take a top-down approach to the situation we firstly discuss the responsibility of the hotel. Intuitively it could be argued that the hotel cannot be held accountable for employing software of which it doesn't know do be faulty. However it can, and will, be held liable for data leaks of personal data if it cannot prove it couldn't have prevented the leak. This is illustrated in Article 23, Chapter 3 of the EU directive on privacy protection and processing of personal data \cite{privacy_directive}, as the hotel will assume the role of the \textit{controller} in this case. The hotel is therefore strictly liable and responsible for the data.

The second potentially responsible party is the software company that developed the product. If we again look at the EU privacy protection directive \cite{privacy_directive} the company could be seen as somewhat of a  \textit{processor}. Even though it doesn't actually perform the processing of personal data in-house, it does supply the software the controller will use to process the data. As stated in Article 17.3, Chapter 2 the same obligations regarding secure processing that apply to the controller are applicable to the processor. Another, more ethical then legal, argument for the responsibility of the software company is the contract of trust between it and the hotel. The hotel trusts the company to develop secure software. If the software company cannot meet the safety expectations it violates the trust placed in it by the hotel. It can therefore be argued that a large part of the responsibility of secure data falls on the shoulders of the software company through transitive / reckless responsibility. A more direct legal and strict liable responsibility of the software company for damage caused by a data leak is presented in the EU directive concerning liablity of defective products \cite{liability_directive}. Article 1 of this directive states:

\begin{displayquote}The producer shall be liable for damage caused by a defective in his product.\end{displayquote}

Psychological damage (Article 9a), such as loss of privacy or sensitive data being leaked, caused to people can be accredited to the producer of the product. The software company should therefore not only be responsible but also feel responsible for damage caused by a fault in their software.
Important to note is that the EU's liability directive \cite{liability_directive} imposes an (implicit) statute of limitations to the liability of the producer. A producer cannot be held accountable for damage caused by faults it could not have foreseen or discovered at the time of production. The degree of responsibility therefore depends heavily on the time that has elapsed between production of the product and discovery of the defect.

Finally we have the individual developer. Legally the developer most likely cannot be held liable for faults in the software. His/her employment by the company protects him/her from that unless intentional misconduct can be proven. However it is not unreasonable for a contributor of a product to feel responsible for the product that it releases and the damage it causes.

\section{Steps taken}
After discussing the different levels of responsibility that can be attributed to which stakeholder, we will delve deeper into a discussion of what we would do in the role of the individual employee. The first step would be to assess the case. This involves an exploratory technological study to the problem and possible solutions. Secondly a risk analysis should be made concerning the likelihood of a data breach/leak and the possible impact this would have on the population. Ideally this estimate should be executed with permission of the management and on the payroll, but even if the bosses don't agree to this a superficial estimate could be made. We now have a broad understanding of the problem, the risk and possible solutions. 

With this information in hand we can go to our superiors and convince them that the benefits of fixing the defect will outweigh the costs. For this we refer back to our discussion on responsibility of the company in which we concluded that the company should feel ethically responsible and can be held liable for the damage it causes by not addressing the defects in their software. Whether this should be done pro bono or for a one time maintenance fee is in the first place up to the company to discuss with the hotel. But the decision will largely depend on the willingness of the company and the complexity of the fix. If the problem is easy and the solution is known the costs should be minimal and the maintenance could be offered for free in order to preserve the good name of the company. In this case one might prefer some unbillable hours over a large class action lawsuit over personal data leaks. However if the problem is new and complicated, and in order to solve it research has to be done, a maintenance fee should be expected.

If the company is completely unwilling in fixing the defect there is one last option for internal resolution. This last option is to encourage the management  to at least notify the hotel (and potential other customers) of the defect. This somewhat absolves the company of their responsibility because it has indicated the defect to the customer and it is up to them to choose as solution, either develop a solution in-house, employ the company again or hire someone else. In either way the proverbial ball is in their court. This option is especially viable when the defect is new in the world of software engineering technology and could not be detected or solved at time of development. If the fix is easy and obvious a message which basically comes down to "It's broken and we won't fix it" might not be appropriate. Note that this option does not absolve the company of clear wrongdoing such as negligence. As a final encouragement for this notification we would argue that when a data leak occurs the company is required by law to notify government instances about this \cite{privacy_directive}. Given this legal requirement we should at least feel the responsibility to inform our customers of such clear and present dangers.

Another case is when the company, given all deliberations, still holds on to the "That'll teach them”-mentality. In this case the company is completely unwilling in co-operating towards an appropriate solution and withdraws itself from the responsibility it has. This case leaves two options: accept the decision of management or reach out to (governmental / privacy) institutions outside the company. The first decision could be seen as lawful as the contract is between the company and the hotel and it is up to them to resolve the conflict. However given that the company is not willing to resolve the defect we still feel a responsibility to do no harm to potentially a large group of people. This is also shown in the ACM code of conducts\cite{acm_code_of_conduct}. Principle 1.04 describes that a software engineer should disclose to the appropriate person any actual or potential danger to the user. Not disclosing would not be allowable by any ethical theory (utilitarian, duty-based or Kantian) except when the loyalty is placed first from a duty-based perspective. We therefore choose the second option and will continue our effort to make the defect known.

The first institution we would visit for this is the hotel itself. As they placed their trust unto us to develop an application capable of securely processing sensitive data we feel they have a right to know they are using an application with potential security leaks. If our company doesn't take the responsibility to notify them of the leak then we will do it. This briefing will be done anonymously in order to not break the trust to our employer. We will also not offer to resolve the defect ourselves as long as we are employed by the software company. This is because it would make us partially responsible for the product in a personal manner instead of just a professional employment manor.

The next option is whistleblowing and reporting the issue to the privacy authority. This will ensure that the defect is registered and instances responsible and potentially liable will be forced to deliberate a solution by law. Important for this is a comprehensive description of action taken by us, and description of actions taken or explicitly not taken by the hotel and the software company, to legitimize our own whistleblowing. It would then be up to the authorities to compel the hotel to comply to privacy protection laws. The software company will probably escape litigation because they are not viewed as the primary data processors (named controller in the literature \cite{privacy_directive}) and will only be addressed when the testimonies indicate gross negligence during the development.

The final action that can be taken is to go public with the information of grave negligence by the hotel and/or the software company. This is an extreme option and should only be taken if a culture of consequent ignorance is present. This is a serious dissatisfaction of the level of professionalism taken by the company and will most likely entail resigning from the software company. However before this action is taken some serious consideration should be taken into account as described by Michael Davis \cite{whistleblowing_paradoxes}:
\begin{enumerate}
\item The defect should entail a clear and present danger to the community. While the defect could cause a large degree of damage depending of the size of a data breach, it still needs to be demonstrated that such a breach is likely or at least conceivable.
\item The defect has been reported to management, but management has indicated that they are not willing to device a solution with or without the involvement of the hotel.
\item All other internal resources have been exhausted and even the hotel itself has proven to be unwilling to mitigate the problem
\end{enumerate}
If the question of probability in the first requirement is answered then it would be permissible to go public with the case. Again this action should only be taken if there is an ongoing tradition of ignorance towards public responsibility present.

All in all, the root of this roadmap of our actions within this theoretical scenario is the well-being of the guests.  We feel obligated to maintain the security of their privacy even if we would lose our employment. If it would come to this, we would not want to work for a company who would act so negligent when we believe they have a moral responsibility to either fix the issue or disclose that there is an issue.
