\chapter{Alternate solution to the whistleblowing case}
\label{chap:alternate_approach}
In this chapter we would like to discuss an alternate method of resolving the situation set in the previous chapter. This alternate approach is led by employee's with different believes of responsibility as is described in the following text.

\section{Approach}
An alternate approach to the case described in the previous chapter is to report the case directly to relevant legal institutions or media. In this case we completely skip the option to resolve the issue from within the company or with our client.
The choice arises from the realization that the attitude of the company towards it responsibility to do no harm will not change. The stubbornness of the company could give rise to the belief that internal resolution will not yield any positive result and may only agitate management creating a tense work environment. 
Secondly, the employee holds no direct professional legal responsibility to the hotel. Its responsibility is to the company, which in turn has a responsibility to the hotel. The responsibility to the hotel does not automatically pass to the employee should the company not honour its responsibility. 
However as programmer on the product an employee can still feel responsible for the damage it can cause. This can cause the employee to want to have the issue in the program resolved, but it feels unable to resolve it by its own. An easy way to do this is to report the issue to legal entities and trust them to resolve the issue by addressing the relevant litigation. 

\section{Discussion}
While an attractive option, this option is the `cowards` way out. It is easy to hand over the information and delegate the problem and resign any personal responsibility. However trying to resolve the issue internally will probably yield much more resolution options than `by the book` legislative options, resulting in a more appropriate resolution.
Secondly, the legislative option might cause unwanted damage to the company. Not only might the misconduct lead to financial penalties but, because these cases are usually handled publicly, the case might leak to the news. This nuances the `do no harm` argument which supported this method and requires another look into which option causes the most harm. Which of these options might attributes the most harm depends on which ethical theory used. Duty-based theory might give cause to the thought that the privacy of persons should be a fundamental right of all humans and should be protected in full. While a more utilitarian view might cause an investigation on what would give the most damage to which people and what the likelihood of the harmful events occurring.
Our final argument against this couse of action taken is that it is simply not supported by the literature. \cite{whistleblowing_paradoxes} states that before any whistle-blowing is ethically allowed one first needs to exhaust every option to resolve the issue internally. Because we just accepted that this would not yield any result and did not even try to do so, we feel that that we did not employ all the internal mechanisms we could have.
Concluding this discussion we feel that the case, literature and theories just don't support this option as an appropriate course of action. We therefore would prefer the option described in chapter \ref{ch:whistleblowing} over the option we described in this chapter.