\chapter{Publicizing habits for survival?}

\section{Problem}
The system described in chapter \ref{chap:system_description} gathers a significant amount of data about individuals. As described in section \ref{sec:extreme_scen} the gathering of the data allows for extreme possibilities e.g. habit analysis per person. Say the hotel is in financial difficulty but there is a solution! A political group believes publicizing the data on the internet is a method to strengthen their point and is offering to buy it. Without selling the data, the hotel will become financially insolvent. All guests signed waiver on their check-ins transferring ownership of said data to the hotel and for the sake of argument there should be no apparent legal issues (in the Netherlands this is ofcourse, impossible). The question that is raised, what should the hotel do? Sell or not?


\section{Why is it an ethical problem?}
We have described the question to sell or not to sell the data as a purely ethical dilemma. The data can contain sensitive, private data regarding habits that might harm (political) careers or show crimes: 1) A prime-minister might have searched for foot fetish video's during his stay as a guest, 2) A celebrity might book a room with an unknown person regularly, indicating a romance, 3) A known drug-dealer is usually checked-in with someone else showing a connection or 4) A specific adult regularly checks-in with a different child companion. The hotel knows the sold data will be publicly released on the internet and it might contain any number of sensitive conclusions about individuals. A similar situation happened recently where Ashley Madison (a website to connect spouses who want to cheat) members were made public through a data-dump of the website. Anyone associated with the website was branded a cheater and members including governmental \& military employees\cite{ashley_madison}.

There are also consequences if the hotel does not sell the data and becomes bankrupt. All of the employees and management will become unemployed which could result in difficulties for any of the employees/management and their families. 

Assuming preventing significant harm to a fellow person is deemed ethical and consciously inflicting it is unethical, we can consider it an ethical problem where either choice will harm a large group of people. The potential harm inflicted is not physical but can be seen as significant harm. Unemployment vs. privacy violation is what is at stake.

While informally we have already shown it is an ethical problem, 

\section{Stakeholders}
Quite a number of parties are affected by either decision:

\begin{itemize}
	\item Guests - Data containing their habits could be made public
	\item Employees - Employees might be laid off if the hotel becomes insolvent
	\item IT Employees - IT Employees might be laid off if the hotel becomes insolvent
	\item Hotel management - Management might be laid off if the hotels becomes insolvent or their reputation harmed if they decide to sell the data
	\item Buyers - They want to further their goals by buying and making the habit data public
\end{itemize}


\section{Responsibility}
The start of this scenario describes that the decision solely lies in the hands of the hotel management. However, they are not the only ones responsible for any of the outcome.

First, the guests. All of the guests had to sign a waiver transferring any data gained through monitoring as property to the hotel. In other words, the guests explicitly consent that anything they do during their stay, might be made public by the hotel. If they did not agree with these terms, they simply should not have signed the waiver and left the hotel. They are responsible for deciding which actions may be monitored by the hotel by either doing those actions during their stay or not.

The employees are responsible for themselves and their guests. They have the guests in their care as per their job description. Also, depending on your views, they are responsible for their own situations to some degree.

The IT employees have the same responsibilities as the other employees with the added responsibility for the system which monitors the guests.

The hotel management is the main responsible party here either way. They are responsible for themselves, the employees, the guests and the deal with the buyers. The decision is in their hands and they should decide what they deem more important.

Lastly, the buyers. They have the controversial goal of publicizing the data to forward their goals. Not looking to closely at the ethical perspective of their goal, they have given the hotel management the choice of selling the data. They are responsible for the creation of the deal and what they will do with the data if the deal is successful.
  
\section{Solution time!}
The stage has been set and the context given. The problem, stakeholders and responsibilities are all clear. While their are many solutions, we shall focus on five distinct ones: 1) Sell the data, 2) Do not sell the data, 3) Analyse all of the data first, 4) Sell the data after anonymizing it and 5) Sell the data partially. Each of these solutions have their own positive and negative consequences. We shall use these consequences to compare the different solutions.  

\subsection{Sell the data}
Our first option is to blatantly sell the data. This would probably save the hotel and the employment of many but there would be no idea of the consequences. Due to the amount of nature of the data it would be possible to analyse certain behaviours for any of the guests in both past and present. While this might lead to interesting facts about your neighbour if they visited the hotel at some point in time, the focus of the public would probably be on any high-profile guests. As stated before, certain crimes might also come to light. Speculating further, this sensitive information might be used for stalking or blackmailing while it could also be used to identify criminals. All in all we will know for sure that the privacy of the previous guests during their stay is gone. A risk here is that the buyers will also announce from whom they purchased the data which might result in guests not visiting the hotel anymore.

\subsection{Do not sell the data}
Our second option is to turn down the offer. The privacy of the guests would be kept save a little longer but we do not know what will happen when the hotel becomes insolvent. The data may be erased or dumped anyway. What is sure is the unemployment of both the management and the employees. Certain employees who were especially dependent on the income might have issues paying their bills. It is also possible not everyone is able to find a new job. 

\subsection{Analyse all of the data first}
While highly impractical, our third option is to analyse the data first to get a clearer picture of the consequences of what would happen if the data leaked. Say the analysis says the data would not contain anything worthwhile and the hotel do decides to sell it. Nothing would stop the buyers from doctoring the data for their views. Their subjectivity and dependency to the data would not guarantee an honest result. Also the analysis is not complete so the buyers might find something the hotel is not able to find.

\subsection{Anonymize and then sell the data}
Our fourth solution is to first anonymize the data to certain degree k. This degree k is defined as how "anonymous" a set of data is by Sweeney\cite{k_anonymity}. The next question would be what to anonymous and how far you should go. Also, there is the risk the buyers would not accept the data in an anonymized form. Depending on how anonymous the data would be and if the buyers would accept it, this might lead to a win-all situation where both the employees and the guests' privacy (to a certain degree) are save.

\subsection{Sell the data partially}
Our fifth solution is to sell the data partially. For analyzing habits, especially the internet metadata, movement data, tv usage and consumption history are important as they provide so much detail to a situation. Knowing the history of stays could show a pattern, but any of the previous mentioned data could provide context to that pattern. Selling the data partially would not anonymize the data but would anonymize possible results to a degree. As with anonymizing the data, it could result in a possible win-all situation if the buyers accept the partial data and if enough data could be skipped.

\subsection{"Best" solution} 
Any of the five previous mentioned options have drawbacks. These are either risk, a partial degree of success in saving the privacy of the guests, insolvency and dishonest results from the buyers. As there is no option without drawbacks, there is no "perfect" solution. A chosen solution would be based completely on the views and beliefs of the hotel management. If they would debate, a more duty-based view could vote for throwing the guests under the bus to remain completely loyal to the employees. A more virtue-based, paternalistic view could vote for the "honest" route by declining the offer and going down with the ship to protect the guests. Lastly, a more utilitarianism approach could be to first analyse the data and to look at the amount of employees. Depending on how much predicted "harm" would be created, a balanced decision might lead to either. The risk however is a false prediction. The buyers might change the data to fit their desired result which might be more catastrophic then the prediction of the harm to the guests' privacy would be. Also if they announce from whom they purchased the data insolvency and publication of data might be the result after-all. Therefore the most utilitarian choose would be to not sell as it is the most deterministic amount of harm.




- Let the employees decide
Code of conduct 2.09
Many more...