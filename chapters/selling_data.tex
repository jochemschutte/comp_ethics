\chapter{Publicizing habits for survival?}
\label{chap:selling_data}

\section{Problem}
The system described in chapter \ref{chap:system_description} gathers a significant amount of data about individuals. As described in section \ref{sec:extreme_scen} the gathering of the data allows for extreme possibilities e.g. habit analysis per person. Say the hotel is in financial difficulty but there is a solution! A political group believes publicizing the data on the internet is a method to strengthen their point and is offering to buy it. Without selling the data, the hotel will become financially insolvent. All guests signed a waiver on their check-ins transferring ownership of said data to the hotel and for the sake of argument there should be no apparent legal issues (in the Netherlands this is of course, impossible). The question that is raised, what should the hotel do? Sell or not?


\section{Why is it an ethical problem?}
We have described the question to sell or not to sell the data as a purely ethical dilemma. The data can contain sensitive, private data regarding habits that might harm (political) careers or show crimes: 1) A prime-minister might have searched for foot fetish video's during his stay as a guest, 2) A celebrity might book a room with an unknown person regularly, indicating a romance, 3) A known drug-dealer is usually checked-in with someone else showing a connection or 4) A specific adult regularly checks-in with a different child companion. The hotel knows the sold data will be publicly released on the internet and it might contain any number of sensitive conclusions about individuals. A similar situation occured recently where Ashley Madison (a website to connect spouses who want to cheat) members were made public through a data-dump of the website. Anyone associated with the website was branded a cheater and members including governmental \& military employees\cite{ashley_madison}. While the stigma is different (website for cheaters vs. a hotel) the type of conclusions drawn from the data can be similar.

There are also consequences if the hotel does not sell the data and goes bankrupt. All of the employees and management will become unemployed which could result in difficulties for any of the employees/management and their families. 

Assuming preventing significant harm to a fellow person is deemed ethical and consciously inflicting it is unethical, we can consider it an ethical problem where either choice will harm a large group of people. The potential harm inflicted is not physical but can be seen as significant harm. Unemployment vs. privacy violation is what is at stake.

While informally we have already shown it is an ethical problem, the literature agrees with us. The ACM code of ethics \cite{acm_code_of_conduct} describes in Principle 1 that software engineers and their employers are should "act consistently with the public interest". More specific, Principle 1.02 states: "Moderate the interests of the software engineer, the employer, the client and the users with the public good." and Principle 2.09 "Promote no interest adverse to their employer or client, unless a higher ethical concern is being compromised;...". Unfortunately, in this situation the management has to decide how to protect both the client's and the employee's interests while it looks as if either choice will harm one of those groups.

\section{Stakeholders}
Quite a number of parties are affected by either decision:

\begin{itemize}
	\item Guests - Data containing their habits could be made public
	\item Employees - Employees might be laid off if the hotel becomes insolvent
	\item IT Employees - IT Employees might be laid off if the hotel becomes insolvent
	\item Hotel management - Management might be laid off if the hotels becomes insolvent or their reputation harmed if they decide to sell the data
	\item Buyers - They want to further their own goals by buying and making the habit data public
\end{itemize}


\section{Responsibility}
The start of this scenario describes that the decision solely lies in the hands of the hotel management. However, they are not the only ones responsible for any of the outcome.

First, the guests. All of the guests had to sign a waiver transferring any data gained through monitoring as property to the hotel. In other words, the guests explicitly consented that anything they do during their stay might be made public by the hotel. If they did not agree with these terms, they simply should not have signed the waiver and left the hotel. They are responsible for deciding which actions may be monitored by the hotel by either doing those actions during their stay or not. As they have given permission for their information to be used in the described way, Principle 2.03 of the ACM Code of Ethics \cite{acm_code_of_conduct} tells us it is ethically allowed. Principle 2.03: "Use the property of the a client or employer only in ways properly authorized, and with the client's or employer's knowledge and consent." What is unknown though is if the guests knew they were giving their consent for their data to be recorded and used. As it was clearly stated in the waiver, the hotel should be able to assume the guests knew what they were signing.

The employees are responsible for themselves and their guests. They have the guests in their care as per their job description. Also, depending on your views, they are responsible for their own situations to some degree. Again, the ACM Code of Ethics \cite{acm_code_of_conduct} agrees here as Principle 2 describes: "Client and Employer shall act in a manner that is in the best interests of their client and employer, consistent with the public interest."

The IT employees have the same responsibilities as the other employees with the added professional responsibility for the system which monitors the guests.

The hotel management is the main responsible party here either way. They are responsible for themselves, the employees, the guests and the deal with the buyers. The decision is in their hands and they should decide what they deem more important.

Lastly, the buyers. They have the controversial goal of publicizing the data to forward their own goals. Not looking to closely at the ethical perspective of their goal, they have given the hotel management the choice of selling the data. They are responsible for the creation of the deal and what they will do with the data if the deal is successful.
  
\section{Solution time!}
The stage has been set and the context given. The problem, stakeholders and responsibilities are all clear. While there are many solutions, we shall focus on five distinct ones: 1) Sell the data, 2) Do not sell the data, 3) Analyze all of the data first, 4) Sell the data after anonymizing it and 5) Sell the data partially. Each of these solutions have their own positive and negative consequences. We shall use these consequences to compare the different solutions.  

\subsection{Sell the data}
Our first option is to blatantly sell the data. This would probably save the hotel and the employment of financial crisis, but there would be no way to predict the consequences of the sale. Due to the amount and the nature of the data it would be possible to analyze certain behaviours for any of the guests in both past and present. In principle this could lead to interesting facts about your neighbours if they visited the hotel at some point in time, the focus of the public would probably be on any high-profile guests. As stated before, certain crimes might also come to light. Speculating further, this sensitive information might be used for stalking or blackmailing while it could also be used to identify criminals. All in all we will know for sure that the privacy of the previous guests during their stay is completely lost. A risk here is that the buyers will also announce from whom they purchased the data which might result in guests not visiting the hotel any-more. Also, even if the buyers do not disclose who supplied the information, the contents of the information is also usable to determine which hotel supplied the data.

\subsection{Do not sell the data}
The other extreme of solution spectrum is the decision to turn down the offer. The privacy of the guests would be kept save a little longer but we do not know what will happen when the hotel becomes insolvent. The data may be erased or dumped anyway. What is sure is the unemployment of both the management and the employees. Certain employees who were especially dependent on the income might have issues paying their bills. It is also possible that not everyone is able to find a new job. 

\subsection{Analyze all of the data first}
While highly impractical, our third option is to analyze the data first to get a clearer picture of the consequences of what would happen if the data leaked. Say the analysis says the data would not contain anything worthwhile and the hotel does decide to sell it. In this case it is important to note that nothing would stop the buyers from doctoring the data for their views. Their subjectivity and dependency to the data would not guarantee an honest result. Also the analysis is not complete so the buyers might find something the hotel is not able to detect. While impractical and not a perfect solution, we will prefer this over solution 1 as the hotel atleast tries to gather more information first to make a more informed decision.

\subsection{Anonymize and then sell the data}
Our fourth solution is to first anonymize the data to certain degree k. This degree k is defined as how "anonymous" a set of data is by Sweeney\cite{k_anonymity}. The next question would be what to anonymous and how far you should go. Also, there is the risk the buyers would not accept the data in an anonymized form. Depending on how anonymous the data would be and if the buyers would accept it, this might lead to a win-all situation where both the employees and the guests' privacy (to a certain degree) are save.

\subsection{Sell the data partially}
Our fifth solution is to sell the data partially. For analyzing habits, especially the internet metadata, movement data, tv usage and consumption history are important as they provide so much detail and context to a situation. Knowing the history of stays could show a pattern, but without this contextual data (internet metadata, movement data, tv uage and consumption history) there would be almost to no context to the pattern. Selling the data partially would not anonymize the data completely (patterns can still be led to individuals) but would anonymize possible results to a degree. Example: Someone visits the hotel regularly each month. What they were doing in the hotel is not in the data as the content of their stay cannot be led from the data. As with solution 4, anonymizing the data, it could result in a possible win-all situation if the buyers accept the partial data and enough contextual data could be removed from the sold data.

\subsection{"Best" solution} 
Any of the five previously mentioned options have drawbacks. These are: 1) risks for the hotel, 2) only a partial degree of success in retaining the privacy of the guests, 3) insolvency and 4) dishonest results from the buyers. As there is no option without drawbacks, there is no "perfect" solution. A chosen solution would be based completely on the views and beliefs of the hotel management.

If they would debate, a more duty-based view could vote for throwing the guests under the bus to remain completely loyal to the employees. A more virtue-based, paternalistic view could vote for the "honest" route by declining the offer and going down with the ship to protect the guests.

Lastly, a more utilitarianism approach would be to create a balanced decision. Balanced here is the amount of harm, risk and profit that would result from the decision. With any solution where the the data is sold, there is the risk the buyers will also publicize who sold the data and/or manipulate the data before publicizing it. The possible sources of harm are violating the privacy of the guests and unemployment for the employees. The ideal solution would be no risk with no harm and the worst solution is all risk with all harm. However, there is a problem with defining how much harm there will be. The sources of harm are closely linked with the risk.

A utilitarian would want to create a clear picture of the amount of harm each decision will bring: 1) Selling the data has the consequence of violating the privacy of the guests completely with the risk of publicizing that the hotel sold the data and the buyers might change the data. 2) Not selling the data has no risk but will result in unemployment. 4) First anonymize the data to a degree k and then sell the data. This has a lower amount of privacy violated but still has the same risk of publicizing the hotel sold the data and the buyers might change the data. 5) Selling the data partially will result in a lower amount of privacy violated but still has the same risk of publicizing the hotel sold the data and the buyers might change the data. Solution 3) again results in a choice but with a clearer picture of how much privacy would be violated.

Using a very simple calculation schema, solution 2 has no risk while all other solution have full risk and will only harm 1 party so it gets 1 point. Solutions 1, 4 and 5 will result in harming 1 party completely or partially but with full risk. Therefore, they range from 2-3 points. Using this very simple calculation schema, the most utilitarian choice is solution 2. But as shown, the most "utilitarian" choice depends completely on the calculation scheme used.

All in all, a best solution would not exist. As we are more duty/virtue-based towards the customers we would choose solution 2 from the proposed solutions and we would not sell the data. We would of course try to safe the company in a more virtuous way without shady deals to sell-out our customers but we would go down with the ship if it came to it.
